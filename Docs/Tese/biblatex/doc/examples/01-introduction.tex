%
% This file briefly presents the main citation commands. It also
% illustrates the typical structure of a document based on biblatex.
%
\documentclass[a4paper]{article}
\usepackage[T1]{fontenc}
\usepackage[utf8]{inputenc}
%
% We load the babel package and specify the main language of the
% document. If babel is detected, biblatex will automatically adjust
% to the default language of the document.
\usepackage[american]{babel}
%
% We load the csquotes package, which is required for
% language-specific quotation marks. See the csquotes manual for
% further information about this package.
\usepackage{csquotes}
%
% We are using the 'authoryear' style in this example.
% The default is the 'numerical' style.
\usepackage[style=authoryear,backend=biber]{biblatex}
%
% We use the database biblatex-examples.bib.
\addbibresource{biblatex-examples.bib}
%
% Some generic settings.
\newcommand{\cmd}[1]{\texttt{\textbackslash #1}}
\setlength{\parindent}{0pt}
\begin{document}

\section*{Standard citation commands}

\subsection*{The \cmd{cite} command}

% The \cite command prints a bare citation without parentheses.

\cite{companion}

\cite[59]{companion}

\cite[see][]{companion}

\cite[see][59--63]{companion}

\subsection*{The \cmd{parencite} command}

% The \parencite command, which is intended for in-text citations,
% encloses the citation in parentheses. Note that the 'numeric' and
% 'alphabetic' styles use square brackets instead.

This is just filler text \parencite{companion}.

This is just filler text \parencite[59]{companion}.

This is just filler text \parencite[see][]{companion}.

This is just filler text \parencite[see][59--63]{companion}.

\subsection*{The \cmd{footcite} command}

% The \footcite command is similar to \parencite, except that the
% citation is given in a footnote.

This is just filler text.\footcite{companion}

This is just filler text.\footcite[59]{companion}

This is just filler text.\footcite[See][]{companion}

This is just filler text.\footcite[See][59--63]{companion}

\subsection*{The \cmd{textcite} command}

% The \textcite command is intended for citations integrated in the
% flow of text, replacing the subject of the sentence.

\textcite{companion} show that this is just filler text.

\textcite[59]{companion} show that this is just filler text.

% With \textcite, the first optional argument is of limited use
% only, since you could simply place the prenote in front of the
% citation. It is still supported for the sake of consistency.

\textcite[see][]{companion} show that this is just filler text.

\textcite[see][59--63]{companion} show that this is just filler text.

\section*{Style-independent commands}

\subsection*{The \cmd{autocite} command}

% The point of the \autocite command is that it automatically adapts
% to the predominant citation format (inline or footnote) normally
% used with the selected citation style. It should be used at the
% end of the sentence and usually works like \parencite or
% \footcite, depending on the citation style and the setting of the
% 'autocite' package option. With the author-year style used in this
% example, it works like \parencite:

This is just filler text \autocite{companion}.

\section*{Text commands}

% There are a few predefined commands for bibliographic data which
% is frequently used in the flow of text. Note that biblatex also
% grants access to all lists and fields at a lower level, see below.

\citeauthor{companion} show that this is just filler text.

% Note that \citetitle will use the 'shorttitle' field, if
% available.

See the \citetitle{companion} for more filler text.

% \citetitle*, on the other hand, always prints the 'title'
% field.

See also: \citetitle*{companion}. It has more filler text.

\citetitle*{companion} by \citeauthor{companion} was
published in \citeyear{companion}.

\section*{Special commands}

% The \nocite command adds an entry to the bibliography without
% printing any citation. If the argument is an asterisk (i.e.,
% '\nocite{*}'), all entries found in the bib file are added to the
% bibliography.

No citation here.\nocite{knuth:ct}

% The \fullcite command prints a verbose citation similar to the
% full bibliography entry.

\fullcite{companion}

% \footfullcite is similar to \fullcite, except that the citation is
% put in a footnote.

This is just filler text.\footfullcite{companion}

\section*{Low-level commands}

% The commands in this section grant low-level access to all
% bibliographic data. See the biblatex manual for details about the
% different data types (name lists, literal lists, fields).

% The \citename command prints a name list. For example, the
% \citeauthor command shown above is similar to the following
% low-level command:

\citename{companion}{author}

% The \citelist command is similar to \citename but prints literal
% lists.

\citelist{companion}{publisher}

% The \citefield command prints fields. For example, the
% \citetitle* command shown above is similar to the following
% low-level command (except for the formatting):

\citefield{companion}{title}

% The list of references is printed by way of \printbibliography.

\printbibliography

\end{document}
