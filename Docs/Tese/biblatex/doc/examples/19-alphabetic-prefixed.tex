%
% Prefixes also work with alphabetic the styles.
%
\documentclass[a4paper,oneside]{article}
\usepackage[T1]{fontenc}
\usepackage[utf8]{inputenc}
\usepackage[american]{babel}
\usepackage{csquotes}
\usepackage[style=alphabetic,defernumbers,backend=biber]{biblatex}
\usepackage{hyperref}
\usepackage{nameref}
\addbibresource{biblatex-examples.bib}
% A catch-all filter for all items which are not assigned to a
% dedicated sub-bibliography:
\defbibfilter{other}{
  not type=article
  and
  not type=book
  and
  not type=collection
}
\begin{document}

\section*{Prefixed alphabetic citations}

% Some citations:
\cite{angenendt, kastenholz, augustine, companion, jaffe, ctan}

\nocite{*}

% Let's print the overall heading of the bibliography first:
\printbibheading

% And now the sub-bibliographies: we use three of them (based on the
% entry type). Each sub-bibliography assigns a different prefix.
\newrefcontext[labelprefix={A-}]
\printbibliography[heading=subbibliography,title={Articles},type=article]
\newrefcontext[labelprefix={B-}]
\printbibliography[heading=subbibliography,title={Books},type=book]
\newrefcontext[labelprefix={C-}]
\printbibliography[heading=subbibliography,title={Collections},type=collection]

% The catch-all sub-bibliography for all remaining types:
\newrefcontext[labelprefix={O-}]
\printbibliography[heading=subbibliography,title={Other Sources},filter=other]

\end{document}
