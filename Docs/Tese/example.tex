%
% uaThesis example (for a thesis written in Portuguese)
%
% the complete list of options and commands can be found in uaThesis.sty
%

\documentclass[11pt,twoside,a4paper]{report}
%\documentclass[11pt,twoside,a4paper,openright]{report}    % Estilo relatório
% openright: página inicial de cada capítulo sempre ímpar
\usepackage[DETI,newLogo]{uaThesis}

\def\ThesisYear{2018}

% optional packages
\usepackage[portuguese]{babel}
\usepackage{hyperref}
\usepackage{amsmath}
\usepackage{amssymb}
\usepackage{xspace}% used by \sigla

\graphicspath{ {Pictures/} }
    % Símbolos da American Mathematical Society
    \usepackage{mathtools}
    \usepackage{amsmath}
    %\usepackage[citebordercolor={1 1 1},linkbordercolor={1 1 1}]{hyperref}
    \usepackage{graphicx} % Required for the inclusion of images
    \usepackage{subcaption}   
    \usepackage{float}
    \usepackage{listings}

%encoding
%--------------------------------------
\usepackage[utf8]{inputenc}
%\usepackage[T1]{fontenc}
%--------------------------------------

% optional (comment to use default)s
%   depth of the table of contents
%     1 ... chapther and sections
%     2 ... chapters, sections, and subsections
%     3 ... chapters, sections, subsections, and subsubsections
\setcounter{tocdepth}{3}

% optional (comment to used default)
%   horizontal line to separate floats (figures and tables) from text
\def\topfigrule{\kern 7.8pt \hrule width\textwidth\kern -8.2pt\relax}
\def\dblfigrule{\kern 7.8pt \hrule width\textwidth\kern -8.2pt\relax}
\def\botfigrule{\kern -7.8pt \hrule width\textwidth\kern 8.2pt\relax}

% custom macros (could also be defined using \newcommand)
\def\I{\mathtt{i}}         % one possible way to represent $\sqrt{-1}$
\def\Exp#1{e^{2\pi\I #1}}  % argument inside braces, i.e., "{}"
\def\EXP#1.{e^{2\pi\I #1}} % argument finishes when a full stop is encountered, i.e., "."
\def\sigla{\LaTeX\xspace}  % use as "blabla \sigla blabla (no need to do "blabla \sigla\ blabla"

\def\AddVMargin#1{\setbox0=\hbox{#1}%
                  \dimen0=\ht0\advance\dimen0 by 2pt\ht0=\dimen0%
                  \dimen0=\dp0\advance\dimen0 by 2pt\dp0=\dimen0%
                  \box0}   % add extra vertical space above and below the argument (#1)
\def\Header#1#2{\setbox1=\hbox{#1}\setbox2=\hbox{#2}%
           \ifdim\wd1>\wd2\dimen0=\wd1\else\dimen0=\wd2\fi%
           \AddVMargin{\parbox{\dimen0}{\centering #1\\#2}}} % put #1 on top #2


\begin{document}

%
% Cover page (use only one of the first two \TitlePage)
%

% First alternative, with a figure
\TitlePage
  %\GRID  % for debugging ONLY
  \HEADER{\BAR\FIG{\includegraphics[height=60mm]{uaLogoOld}}} % the \FIG{} is optional
         {\ThesisYear}
  \TITLE{Bruno Manuel \newline de Moura Ramos}
        {Sistema de Recolha e Armazenamento Remoto de Informação Sensorial de um Processo Industrial usando Bases de Dados Múltiplas}
\EndTitlePage
\titlepage\ \endtitlepage % empty page


\TitlePage
  \vspace*{55mm}
  \TEXT{\textbf{o juri/the jury\newline}}
       {}
  \TEXT{presidente/president}
       {\textbf{ABC}\newline {\small
        Professor Catedratico da Universidade de Aveiro (por delegacao da Reitora da
        Universidade de Aveiro)}}
  \vspace*{5mm}
  \TEXT{vogais/examiners committee}
       {\textbf{DEF}\newline {\small
        Professor Catedratico da Universidade de Aveiro (orientador)}}
  \vspace*{5mm}
  \TEXT{}
       {\textbf{GHI}\newline {\small
        Professor associado da Universidade J (co-orientador)}}
  \vspace*{5mm}
  \TEXT{}
       {\textbf{KLM}\newline {\small
        Professor Catedratico da Universidade N}}
\EndTitlePage
\titlepage\ \endtitlepage % empty page

\TitlePage
  \vspace*{55mm}
  \TEXT{\textbf{agradecimentos~/\newline acknowledgements}}
       {ergergerg}
  \TEXT{}
       {ergergerg}
\EndTitlePage
\titlepage\ \endtitlepage % empty page

\TitlePage
  \vspace*{55mm}
  \TEXT{\textbf{Resumo}}
       {ergergergerg}
  \TEXT{}
       {bergergerg}
\EndTitlePage
\titlepage\ \endtitlepage % empty page

\TitlePage
  \vspace*{55mm}
  \TEXT{\textbf{Abstract}}
       {Nowadays, it is usual to evaluate a work \ldots}
\EndTitlePage
\titlepage\ \endtitlepage % empty page


%
% Tables of contents, of figures, ...
%

\pagenumbering{roman}
\tableofcontents

\cleardoublepage
\listoffigures

\cleardoublepage
\listoftables


% The chapters (usually written using the isolatin font encoding ...)

\cleardoublepage
\pagenumbering{arabic}
\chapter{Introdução}
arquivo e monitorização de moldes.


\cleardoublepage
\chapter{Estado de Arte}

\cleardoublepage
\chapter{Proposta de Solução}
\section{Infraestrutura de dados}

\section{Base de Dados}
\subsection{Análise de Requisitos}

\subsection{Desenho conceptual e esquema lógico}

\subsection{Construção da base de dados}

\subsection{Programa de transferência}

\subsection{Gestão de \textit{backups}}

\subsection{Simulador}

\subsection{Utilizadores}

\cleardoublepage
\chapter{Aplicação}
A aplicação desenvolvida em ambiente  \textit{Web} com o objetivo de ser multiplataforma, permitir acesso remoto com ligação à \textit{internet} e sem necessitar de instalar \textit{softwares} nos dispositivos dos utilizadores. Esta corre num servidor \textit{Apache} e foi desenvolvida usando \textit{PHP} e \textit{HTML}. Este capítulo descreve a adaptação da infraestrutura desenvolvida e as várias funcionalidades da aplicação.

\section{Adaptação da infraestrutura}
Afim de garantir uma maior integridade dos dados inseridos pela aplicação, instala-se no servidor local uma nova base de dados temporária local. Aqui os utilizadores têm a liberdade para adicionar, alterar e apagar informação sem consequências no sistema para depois serem introduzidas nas bases de dados central e local como representado na \autoref{fig:adap1}. Como referido anteriormente, esta base de dados difere das restantes, não contendo em si as tabelas fase e registos.
\begin{figure}[H]
	\begin{center}
		\includegraphics[width=0.65\textwidth]{placeholder} % Include the image placeholder.png
		\caption{Esquema ligação temporária}
		\label{fig:adap1}
	\end{center}
\end{figure}

\section{Interface gráfica}
A aplicação divide-se em cinco partes distintas:
\begin{itemize}
	\item \textit{Main}
	\item \textit{Login}
	\item Consultas
	\item Administração
	\item Conexão Local
\end{itemize}
A aplicação foi realizada com vista a uma utilização geral e local. A primeira visa um uso a partir de qualquer dispositivo e acessível a qualquer momento e a segunda foca-se num acesso local com o objetivo de configurar e definir a informação no servidor local. As páginas Main, Login, Consultas e parte das funcionalidades da Administração foram realizadas para uma utilização geral. As páginas Conexão Local e as restantes funcionalidades da Administração foram realizadas para uma utilização local.

\subsection{\textit{Main}}
\begin{figure}[H]
\centering
	\begin{minipage}{.5\textwidth}
		\begin{center}
			\includegraphics[width=0.9\textwidth]{placeholder} % Include the image placeholder.png
			\subcaption{Main sem Login}
			\label{fig:main1}
		\end{center}
	\end{minipage}%
	\begin{minipage}{.5\textwidth}
		\begin{center}
			\includegraphics[width=0.9\textwidth]{placeholder} % Include the image placeholder.png
			\subcaption{Main com Login}
			\label{fig:main2}
		\end{center}
	\end{minipage}
	\caption{Main}
	\label{fig:main0}
\end{figure}
A página \textit{Main} serve como página principal da aplicação. Se não houver sessão iniciada todas as restantes páginas redirecionam o utilizador para aqui. Contém apenas algumas informações gerais sobre a aplicação. Iniciar sessão na página de \textit{Login} desbloqueia funcionalidades na aplicação, como demonstrado nas Figuras \ref{fig:main1} e \ref*{fig:main2}. Depois de iniciada opção é possível com os botões navegar para as páginas de Consultas, Administração e Conexão Local.

\subsection{\textit{Login}}
\begin{figure}[H]
	\begin{center}
		\includegraphics[width=0.65\textwidth]{placeholder} % Include the image placeholder.png
		\caption{Esquema ligação temporária}
		\label{fig:login0}
	\end{center}
\end{figure}
A página de \textit{Login} consiste num simples formulário constituído por duas caixas de texto e um botão, como demonstrado na \autoref*{fig:login0}. O botão \textit{Login} lê as credenciais introduzidas e realiza uma conexão de teste à base de dados central, validando a informação introduzida diretamente com \textit{MySQL}. Se as credenciais forem validadas com sucesso o utilizador é redirecionado para a página principal. Se as credenciais introduzidas não forem suficientes ou válidas são retornados erros de forma a informar o utilizador como demonstrado nas Figuras \ref{fig:login2} e \ref{fig:login3}.
\begin{figure}[H]
	\centering
	\begin{minipage}{.5\textwidth}
		\begin{center}
			\includegraphics[width=0.9\textwidth]{placeholder} % Include the image placeholder.png
			\subcaption{Main sem Login}
			\label{fig:login2}
		\end{center}
	\end{minipage}%
	\begin{minipage}{.5\textwidth}
		\begin{center}
			\includegraphics[width=0.9\textwidth]{placeholder} % Include the image placeholder.png
			\subcaption{Main com Login}
			\label{fig:login3}
		\end{center}
	\end{minipage}
	\caption{Main}
	\label{fig:login1}
\end{figure}

\subsection{Consultas}

\subsection{Administração}

\subsection{Conexão local}


\cleardoublepage
\chapter{Instalação do Sistema}

\cleardoublepage
\chapter{Conclusões}


%
% The bibliography
%
\cleardoublepage
\iffalse
  % Use this is the final version
  %  unsrt produces numbered entries, sorted by order of citation
  %  plain produces numbered entries, sorted alphabetically
  %  other styles are possible (I recommend the harvard package)
  \bibliographystyle{unsrt}
  %\bibliographystyle{plain}
  \bibliography{my-bib-file}% replace by the name of name of your .bib file
\else
  % An example (the contents of the .bbl file)
  \begin{thebibliography}{10}

  \bibitem{Eliahou-1-1993-CLBNCL}
  Shalom Eliahou.
  \newblock The $3x+1$ problem: New lower bounds on nontrivial cycle lengths.
  \newblock {\em Discrete Mathematics}, 118(1--3):45--56, 1993.

  \bibitem{Garner-1981-1-OCA}
  Lynn~E. Garner.
  \newblock On the collatz $3n+1$ algorithm.
  \newblock {\em Proceedings of the American Mathematical Society}, 82(1):19--22,
    May 1981.
  \end{thebibliography}
\fi
\cleardoublepage

\end{document}
