%
% uaThesis example (for a thesis written in Portuguese)
%
% the complete list of options and commands can be found in uaThesis.sty
%

\documentclass[11pt,twoside,a4paper]{report}
%\documentclass[11pt,twoside,a4paper,openright]{report}    % Estilo relatório
% openright: página inicial de cada capítulo sempre ímpar
\usepackage[DETI,newLogo]{uaThesis}

\def\ThesisYear{2018}

% optional packages
\usepackage[portuguese]{babel}
\usepackage{hyperref}
\usepackage{amsmath}
\usepackage{amssymb}
\usepackage{xspace}% used by \sigla

\graphicspath{ {Pictures/} }
    % Símbolos da American Mathematical Society
    \usepackage{mathtools}
    \usepackage{amsmath}
    %\usepackage[citebordercolor={1 1 1},linkbordercolor={1 1 1}]{hyperref}
    \usepackage{graphicx} % Required for the inclusion of images
    \usepackage{subcaption}   
    \usepackage{float}
    \usepackage{listings}
    
    \usepackage{enumitem}

%encoding
%--------------------------------------
\usepackage[utf8]{inputenc}
%\usepackage[T1]{fontenc}
%--------------------------------------

% optional (comment to use default)s
%   depth of the table of contents
%     1 ... chapther and sections
%     2 ... chapters, sections, and subsections
%     3 ... chapters, sections, subsections, and subsubsections
\setcounter{tocdepth}{3}

% optional (comment to used default)
%   horizontal line to separate floats (figures and tables) from text
\def\topfigrule{\kern 7.8pt \hrule width\textwidth\kern -8.2pt\relax}
\def\dblfigrule{\kern 7.8pt \hrule width\textwidth\kern -8.2pt\relax}
\def\botfigrule{\kern -7.8pt \hrule width\textwidth\kern 8.2pt\relax}

% custom macros (could also be defined using \newcommand)
\def\I{\mathtt{i}}         % one possible way to represent $\sqrt{-1}$
\def\Exp#1{e^{2\pi\I #1}}  % argument inside braces, i.e., "{}"
\def\EXP#1.{e^{2\pi\I #1}} % argument finishes when a full stop is encountered, i.e., "."
\def\sigla{\LaTeX\xspace}  % use as "blabla \sigla blabla (no need to do "blabla \sigla\ blabla"

\def\AddVMargin#1{\setbox0=\hbox{#1}%
                  \dimen0=\ht0\advance\dimen0 by 2pt\ht0=\dimen0%
                  \dimen0=\dp0\advance\dimen0 by 2pt\dp0=\dimen0%
                  \box0}   % add extra vertical space above and below the argument (#1)
\def\Header#1#2{\setbox1=\hbox{#1}\setbox2=\hbox{#2}%
           \ifdim\wd1>\wd2\dimen0=\wd1\else\dimen0=\wd2\fi%
           \AddVMargin{\parbox{\dimen0}{\centering #1\\#2}}} % put #1 on top #2


\begin{document}

%
% Cover page (use only one of the first two \TitlePage)
%

% First alternative, with a figure
\TitlePage
  %\GRID  % for debugging ONLY
  \HEADER{\BAR\FIG{\includegraphics[height=60mm]{uaLogoOld}}} % the \FIG{} is optional
         {\ThesisYear}
  \TITLE{Bruno Manuel \newline de Moura Ramos}
        {Sistema de Recolha e Armazenamento Remoto de Informação Sensorial de um Processo Industrial usando Bases de Dados Múltiplas}
\EndTitlePage
\titlepage\ \endtitlepage % empty page


\TitlePage
  \vspace*{55mm}
  \TEXT{\textbf{o juri/the jury\newline}}
       {}
  \TEXT{presidente/president}
       {\textbf{ABC}\newline {\small
        Professor Catedratico da Universidade de Aveiro (por delegacao da Reitora da
        Universidade de Aveiro)}}
  \vspace*{5mm}
  \TEXT{vogais/examiners committee}
       {\textbf{DEF}\newline {\small
        Professor Catedratico da Universidade de Aveiro (orientador)}}
  \vspace*{5mm}
  \TEXT{}
       {\textbf{GHI}\newline {\small
        Professor associado da Universidade J (co-orientador)}}
  \vspace*{5mm}
  \TEXT{}
       {\textbf{KLM}\newline {\small
        Professor Catedratico da Universidade N}}
\EndTitlePage
\titlepage\ \endtitlepage % empty page

\TitlePage
  \vspace*{55mm}
  \TEXT{\textbf{agradecimentos~/\newline acknowledgements}}
       {ergergerg}
  \TEXT{}
       {ergergerg}
\EndTitlePage
\titlepage\ \endtitlepage % empty page

\TitlePage
  \vspace*{55mm}
  \TEXT{\textbf{Resumo}}
       {ergergergerg}
  \TEXT{}
       {bergergerg}
\EndTitlePage
\titlepage\ \endtitlepage % empty page

\TitlePage
  \vspace*{55mm}
  \TEXT{\textbf{Abstract}}
       {Nowadays, it is usual to evaluate a work \ldots}
\EndTitlePage
\titlepage\ \endtitlepage % empty page


%
% Tables of contents, of figures, ...
%

\pagenumbering{roman}
\tableofcontents

\cleardoublepage
\listoffigures

\cleardoublepage
\listoftables


% The chapters (usually written using the isolatin font encoding ...)

\cleardoublepage
\pagenumbering{arabic}
\chapter{Introdução}
arquivo e monitorização de moldes.


\cleardoublepage
\chapter{Estado de Arte}

\cleardoublepage
\chapter{Proposta de Solução}
\section{Infraestrutura de dados}

\section{Base de Dados}
\subsection{Análise de Requisitos}

\subsection{Desenho conceptual e esquema lógico}

\subsection{Construção da base de dados}

\subsection{Programa de transferência}

\subsection{Gestão de \textit{backups}}

\subsection{Simulador}

\subsection{Utilizadores}

\cleardoublepage
\chapter{Aplicação}
Aplicação desenvolvida em ambiente  \textit{Web} com o objetivo de ser multiplataforma, permitir acesso remoto e sem recorrer a instalação de \textit{softwares} nos dispositivos dos utilizadores. Esta corre num servidor \textit{Apache} e foi desenvolvida com \textit{PHP} e \textit{HTML}. Este capítulo descreve a adaptação da infraestrutura desenvolvida e as várias funcionalidades da aplicação.\\
\\
CAPAZ DE CONSULTAR E INTRODUZIR VALORES NAS BASES DE DADOS

\section{Adaptação da infraestrutura}
Afim de garantir uma maior integridade dos dados inseridos pela aplicação, instala-se no servidor local uma nova base de dados temporária local. Aqui os utilizadores têm a liberdade para adicionar, alterar e apagar informação sem consequências no sistema antes destas serem introduzidas nas bases de dados central e local como representado na \autoref{fig:adap1}. Como referido anteriormente, esta base de dados difere das restantes, não contendo em si as tabelas fase e registos.
\begin{figure}[H]
	\begin{center}
		\includegraphics[width=0.65\textwidth]{placeholder} % Include the image placeholder.png
		\caption{Esquema ligação temporária}
		\label{fig:adap1}
	\end{center}
\end{figure}

\section{Interface gráfica}
A aplicação divide-se em cinco partes distintas:
\begin{itemize}[noitemsep]
	\item Main
	\item Login
	\item Consultas
	\item Administração
	\item Conexão Local
\end{itemize}
As páginas Main, Login, Consultas e parte das funcionalidades da Administração foram realizadas para uma utilização geral. As páginas Conexão Local e as restantes funcionalidades da Administração foram realizadas para uma utilização local. A primeira visa um uso a partir de qualquer dispositivo e acessível a qualquer momento e a segunda foca-se num acesso local com o objetivo de configurar e definir a informação no servidor local. Por outras palavras, para o utilizador usar as funcionalidades destas páginas tem de aceder à aplicação no sistema local que se situa no cliente.\\
Instalar um molde é culminar de um projeto de elevada responsabilidade, esta ideia junto com a criação da base de dados temporária local serve para melhorar a qualidade da informação introduzida no sistema e diminuir as falhas.\\

\subsection{\textit{Main}}
\begin{figure}[H]
\centering
	\begin{minipage}{.5\textwidth}
		\begin{center}
			\includegraphics[width=0.9\textwidth]{placeholder} % Include the image placeholder.png
			\subcaption{Main sem Login}
			\label{fig:main1}
		\end{center}
	\end{minipage}%
	\begin{minipage}{.5\textwidth}
		\begin{center}
			\includegraphics[width=0.9\textwidth]{placeholder} % Include the image placeholder.png
			\subcaption{Main com Login}
			\label{fig:main2}
		\end{center}
	\end{minipage}
	\caption{Main}
	\label{fig:main0}
\end{figure}
\textit{Main} serve como página principal da aplicação. Se não houver sessão iniciada todas as restantes páginas redirecionam o utilizador para aqui. Contém apenas algumas informações gerais sobre a aplicação.\\
Iniciar sessão na página de \textit{Login} desbloqueia funcionalidades na aplicação, como demonstrado nas Figuras \ref{fig:main1} e \ref{fig:main2}. Depois de iniciada sessão navega-se com os botões para as páginas de Consultas, Administração e Conexão Local.

\subsection{\textit{Login}}
\begin{figure}[H]
	\begin{center}
		\includegraphics[width=0.65\textwidth]{placeholder} % Include the image placeholder.png
		\caption{Esquema ligação temporária}
		\label{fig:login0}
	\end{center}
\end{figure}
A página de \textit{Login} consiste num simples formulário constituído por duas caixas de texto e um botão, como demonstrado na \autoref*{fig:login0}. O botão \textit{Login} lê as credenciais introduzidas e realiza uma conexão de teste à base de dados central validando-as diretamente com \textit{MySQL}. Se as credenciais forem validadas com sucesso redireciona-se o utilizador para a página principal e altera-se o botão de \textit{Login} para \textit{Logout}. Se as credenciais introduzidas não forem suficientes ou válidas são retornados erros de forma a informar o utilizador como demonstrado nas Figuras \ref{fig:login2} e \ref{fig:login3}.\\
Quando se acede à página como \textit{Logout} termina-se a sessão e redireciona-se o utilizador para a página principal.
\begin{figure}[H]
	\centering
	\begin{minipage}{.5\textwidth}
		\begin{center}
			\includegraphics[width=0.9\textwidth]{placeholder} % Include the image placeholder.png
			\subcaption{Main sem Login}
			\label{fig:login2}
		\end{center}
	\end{minipage}%
	\begin{minipage}{.5\textwidth}
		\begin{center}
			\includegraphics[width=0.9\textwidth]{placeholder} % Include the image placeholder.png
			\subcaption{Main com Login}
			\label{fig:login3}
		\end{center}
	\end{minipage}
	\caption{Main}
	\label{fig:login1}
\end{figure}

\subsection{Consultas}
\begin{figure}[H]
	\centering
	\begin{minipage}{.5\textwidth}
		\begin{center}
			\includegraphics[width=0.9\textwidth]{placeholder} % Include the image placeholder.png
			\subcaption{Main sem Login}
			\label{fig:consultas1}
		\end{center}
	\end{minipage}%
	\begin{minipage}{.5\textwidth}
		\begin{center}
			\includegraphics[width=0.9\textwidth]{placeholder} % Include the image placeholder.png
			\subcaption{Main com Login}
			\label{fig:consultas2}
		\end{center}
	\end{minipage}
	\caption{Main}
	\label{fig:consultas0}
\end{figure}
A página de Consultas assiste utilizadores sem conhecimentos de \textit{SQL} a criarem \textit{queries} para consultar a base de dados central. Na \autoref{fig:consultas1} observa-se várias \textit{checkboxes} e três caixas de texto. As \textit{checkboxes} permitem selecionar os atributos que se desejam consultar na base de dados, estes são guardados numa variável @atributos. Quando  se prime o botão \textit{Query} gera-se uma das seguintes \textit{queries}:\\
\begin{lstlisting}[language = SQL]
	SELECT @atributos
	FROM clientes;
	
	SELECT @atributos
	FROM clientes
	INNER JOIN moldes ON cl_ID = m_IDCliente;
	
	SELECT @atributos
	FROM clientes
	INNER JOIN moldes ON cl_ID = m_IDCliente
	INNER JOIN sensores ON m_ID = s_IDMolde
	INNER JOIN tipo ON s_tipo = tipo_ID;
	
	SELECT @atributos FROM clientes
	INNER JOIN moldes ON cl_ID = m_IDCliente
	INNER JOIN sensores ON m_ID = s_IDMolde 
	INNER JOIN tipo ON s_tipo = tipo_ID
	INNER JOIN registos ON s_IDMolde = r_IDMolde
	AND s_num = r_numSensor
	INNER JOIN fase ON r_fase = fase_ID;
\end{lstlisting}
Efetua-se esta seleção com base na coluna mais à esquerda a que os atributos pertencem. Explicando melhor com um exemplo: se o utilizador desejar consultar o cl\texttt{\char`_}ID e o cl\texttt{\char`_}nome da tabela clientes gera-se a primeira \textit{query} no entanto, se o utilizador desejar consultar os atributos cl\texttt{\char`_}ID, m\texttt{\char`_}ID e s\texttt{\char`_}num gera-se a terceira \textit{query}.\\
Além destas, existem três \textit{queries} especificas quando os atributos tipo\texttt{\char`_}nome, fase\texttt{\char`_}nome e r\texttt{\char`_}data\texttt{\char`_}hora são selecionados sozinhos. As primeiras duas permitem consultar as opções disponíveis nos dicionários e a terceira devolve as datas e horas entre o primeiro e último registos.\\
As caixas de texto Filtros e Ordem permitem adicionar às \textit{queries} geradas as cláusulas WHERE e ORDER BY, respetivamente. Para os utilizadores com conhecimentos em \textit{SQL} está disponibilizada a caixa de texto \textit{Query} que permite a criação direta de uma \textit{query}. Este campo está limitado apenas para \textit{queries} do tipo SELECT.\\
Depois da \textit{query} ser gerada retorna-se uma resposta num novo separador como demonstrado na \autoref{fig:consultas2}. O \textit{link} deste resposta contém toda a informação da \textit{query} gerada. Este pode ser arquivado ou enviado para outro utilizador sem ser necessário gerar a \textit{query} novamente, isto é útil para \textit{queries} com muitas cláusulas.\\
Se a \textit{query} não for válida retorna-se um erro de forma a informar o utilizador, como demonstrado nas Figuras \ref{fig:consultas4}, \ref{fig:consultas5} e \ref{fig:consultas6}.
\begin{figure}[H]
	\centering
	\begin{minipage}{.5\textwidth}
		\begin{center}
			\includegraphics[width=0.9\textwidth]{placeholder} % Include the image placeholder.png
			\subcaption{Main sem Login}
			\label{fig:consultas4}
		\end{center}
	\end{minipage}%
	\begin{minipage}{.5\textwidth}
		\begin{center}
			\includegraphics[width=0.9\textwidth]{placeholder} % Include the image placeholder.png
			\subcaption{Main com Login}
			\label{fig:consultas5}
		\end{center}
	\end{minipage}
	\begin{minipage}{.5\textwidth}
		\begin{center}
			\includegraphics[width=0.9\textwidth]{placeholder} % Include the image placeholder.png
			\subcaption{Main com Login}
			\label{fig:consultas6}
		\end{center}
	\end{minipage}
	\caption{Main}
	\label{fig:consultas3}
\end{figure}

\subsection{Administração}
\begin{figure}[H]
	\centering
	\begin{minipage}{.5\textwidth}
		\begin{center}
			\includegraphics[width=0.9\textwidth]{placeholder} % Include the image placeholder.png
			\subcaption{Main sem Login}
			\label{fig:admin1}
		\end{center}
	\end{minipage}%
	\begin{minipage}{.5\textwidth}
		\begin{center}
			\includegraphics[width=0.9\textwidth]{placeholder} % Include the image placeholder.png
			\subcaption{Main com Login}
			\label{fig:admin2}
		\end{center}
	\end{minipage}
	\caption{Main}
	\label{fig:admin0}
\end{figure}
A área de Administração permite ao utilizador alterar informações sobre os clientes, moldes e sensores. A partir de qualquer dispositivo só é possível aceder à Gestão de Clientes como demonstrado na \autoref{fig:admin1}. Nesta área a informação dos clientes pode ser alterada com o formulário demonstrado na \autoref{fig:admin3}. Os botões Adicionar Cliente, Alterar Cliente e Eliminar Cliente executam \textit{queries} do tipo INSERT, UPDATE e DELETE, respetivamente.\\
\begin{figure}[H]
	\begin{center}
		\includegraphics[width=0.65\textwidth]{placeholder} % Include the image placeholder.png
		\caption{Esquema ligação temporária}
		\label{fig:admin3}
	\end{center}
\end{figure}
Como referido anteriormente a aplicação divide-se numa utilização geral e local, todas as funcionalidades descritas até agora têm em vista uma utilização geral. As restantes funcionalidades que são descritas até ao final do capítulo visam um uso local.\\
Após uma conexão bem sucedida ao servidor local do cliente são desbloqueadas novas áreas de gestão como mostra a \autoref{fig:admin2}. As áreas de Gestão de Moldes e Gestão de Sensores demonstradas nas Figuras \ref{fig:admin6} e \ref{fig:admin7}, permitem ao utilizador criar e apagar moldes e sensores, respetivamente. Estes dados são inseridos na base de dados temporária local, aqui o utilizador pode criar e apagar moldes e sensores sem afetar o sistema. Desta forma é possível confirmar a informação introduzida antes de a inserir no sistema. Os botões de Criar e Apagar nestes formulários realizam \textit{queries} do tipo INSERT e DELETE, respetivamente.\\
Quando a informação dos moldes e sensores estiver completa o botão Validar tenta registar os valores presentes na base de dados temporária local nas bases de dados central e local. Se a ação não executar com sucesso é retornado um erro \textit{MySQL} de forma a informar o utilizador. Se a ação executar com sucesso a base de dados temporária local é limpa e os valores são registados permanentemente nas bases de dados central e local.\\
Depois de inseridos, moldes e sensores, não podem ser eliminados via aplicação. Esta opção foi removida da aplicação para evitar erros, dado que apagar um molde em funcionamento faz com que se percam novos registos.\\
Voltando a área de Gestão Clientes, após a conexão local, desbloqueia-se uma nova funcionalidade como demonstra a \autoref{fig:admin5}. O botão Atualizar permite reiniciar o programa de transferência de valores para que este atualize o número de clientes. Com o comando:\\
grep\\
Obtém-se os números de processo dos programas que estão a transferir valores. Estes valores são armazenados na variável @pids. Para terminar os programas utiliza-se o seguinte comando:\\
kill -2 @pids\\
A opção -2 permite enviar para o processo escolhido o sinal SIGINT que é o sinal esperado pelo programa para que este termine as suas rotinas antes de encerrar. Para iniciar novamente o comando usar:\\
~/path/transferencia
\begin{figure}[H]
	\centering
	\begin{minipage}{.5\textwidth}
		\begin{center}
			\includegraphics[width=0.9\textwidth]{placeholder} % Include the image placeholder.png
			\subcaption{Main sem Login}
			\label{fig:admin5}
		\end{center}
	\end{minipage}%
	\begin{minipage}{.5\textwidth}
		\begin{center}
			\includegraphics[width=0.9\textwidth]{placeholder} % Include the image placeholder.png
			\subcaption{Main com Login}
			\label{fig:admin6}
		\end{center}
	\end{minipage}
	\begin{minipage}{.5\textwidth}
		\begin{center}
			\includegraphics[width=0.9\textwidth]{placeholder} % Include the image placeholder.png
			\subcaption{Main com Login}
			\label{fig:admin7}
		\end{center}
	\end{minipage}
	\caption{Main}
	\label{fig:admin4}
\end{figure}
Nas várias áreas de gestão existem os botões Ver Clientes, Ver Moldes e Ver Sensores que executam respetivamente as \textit{queries}:
\begin{lstlisting}[language = SQL]
	SELECT cl_ID, cl_nome, cl_morada, cl_IP, cl_port,
	COUNT(DISTINCT m_ID), COUNT(DISTINCT s_IDMolde, s_num)
	FROM clientes
	LEFT OUTER JOIN moldes ON cl_ID = m_IDCliente
	LEFT OUTER JOIN sensores ON m_ID = s_IDMolde
	GROUP BY cl_ID
	ORDER BY cl_ID
	
	SELECT m_IDCliente, m_ID, m_nome, m_descricao,
	COUNT(DISTINCT s_IDMolde, s_num)
	FROM clientes
	INNER JOIN moldes ON cl_ID = m_IDCliente
	LEFT OUTER JOIN sensores ON m_ID = s_IDMolde
	GROUP BY m_ID
	ORDER BY m_IDCliente, m_ID
	
	SELECT m_IDCliente, s_IDMolde, s_num, tipo_nome,
	s_nome, s_descricao
	FROM moldes
	INNER JOIN sensores ON m_ID = s_IDMolde
	INNER JOIN tipo ON s_tipo = tipo_id
	ORDER BY m_IDCliente, s_IDMolde, s_num
\end{lstlisting}
Estas fornecem algumas informações contextuais para facilitar a navegação do utilizador.

\subsection{Conexão local}
\begin{figure}[H]
	\centering
	\begin{minipage}{.5\textwidth}
		\begin{center}
			\includegraphics[width=0.9\textwidth]{placeholder} % Include the image placeholder.png
			\subcaption{Main sem Login}
			\label{fig:local1}
		\end{center}
	\end{minipage}%
	\begin{minipage}{.5\textwidth}
		\begin{center}
			\includegraphics[width=0.9\textwidth]{placeholder} % Include the image placeholder.png
			\subcaption{Main com Login}
			\label{fig:local2}
		\end{center}
	\end{minipage}
	\begin{minipage}{.5\textwidth}
		\begin{center}
			\includegraphics[width=0.9\textwidth]{placeholder} % Include the image placeholder.png
			\subcaption{Main com Login}
			\label{fig:local3}
		\end{center}
	\end{minipage}
	\caption{Main}
	\label{fig:local0}
\end{figure}
A área de Conectar Local na \autoref{fig:local1} permite realizar uma conexão à base de dados local no servidor do cliente. Com recurso à \textit{query}:
\begin{lstlisting}[language = SQL]
	SHOW DATABASES
\end{lstlisting}
Obtém-se todas as bases de dados instaladas no servidor local. Do ponto de vista prático, cada cliente só terá uma base de dados local mas, para efeitos de desenvolvimento do projeto adotou-se esta vertente.\\
O botão Conectar inicia sessão na base de dados local escolhida e redireciona o utilizador para a página principal como se observar na \autoref{fig:local4}. O botão Desconectar termina esta sessão e redireciona o utilizador também, para a página principal.
\begin{figure}[H]
	\begin{center}
		\includegraphics[width=0.65\textwidth]{placeholder} % Include the image placeholder.png
		\caption{Esquema ligação temporária}
		\label{fig:local4}
	\end{center}
\end{figure}
A área Criar Local na \autoref{fig:local2} permite instalar uma base de dados para um novo cliente. São considerados novos clientes todos os que não tenham moldes associados a si, esta informação obtém-se com a \textit{query}:
\begin{lstlisting}[language = SQL]
	SELECT cl_ID, cl_nome, cl_morada, cl_IP, cl_port
	FROM
		(SELECT cl_ID, cl_nome, cl_morada, cl_IP, cl_port,
		COUNT(DISTINCT m_ID) AS n_moldes
		FROM clientes
		LEFT OUTER JOIN moldes ON cl_ID = m_IDCliente
		GROUP BY cl_ID) AS contagem
	WHERE n_moldes = 0
\end{lstlisting}
Escolhendo um cliente válido o botão Criar cria a base de dados local com as respetivas tabelas e gera ainda as \textit{queries} observadas na \autoref{fig:local5}.
\begin{figure}[H]
	\begin{center}
		\includegraphics[width=0.65\textwidth]{placeholder} % Include the image placeholder.png
		\caption{Esquema ligação temporária}
		\label{fig:local5}
	\end{center}
\end{figure}
Terminado a análise das funcionalidades da aplicação com a área de Instalar \textit{MySQL} na \autoref{fig:local3} que contém os passos para instalar o \textit{MySQL} num sistema \textit{Linux}.

\cleardoublepage
\chapter{Instalação do Sistema}

\cleardoublepage
\chapter{Conclusões}
\section{Observações sobre Infraestrutura}
objetivos concluidos, consumo elevado de processador

\section{Observações Aplicação}
objetivos concluidos, baixa performance por causa do php.\\
sugerir alterar para java e o metodo de instalação.

\section{Observações gerais}
Apesar de algumas quebras de performance o sistema está pronto a ser usado. So tem de ser denvolvido um programa para inserir a query especifica.

\section{trabalhos futuros}
Make over e sistema de notificações.

%
% The bibliography
%
\cleardoublepage
\iffalse
  % Use this is the final version
  %  unsrt produces numbered entries, sorted by order of citation
  %  plain produces numbered entries, sorted alphabetically
  %  other styles are possible (I recommend the harvard package)
  \bibliographystyle{unsrt}
  %\bibliographystyle{plain}
  \bibliography{my-bib-file}% replace by the name of name of your .bib file
\else
  % An example (the contents of the .bbl file)
  \begin{thebibliography}{10}

  \bibitem{Eliahou-1-1993-CLBNCL}
  Shalom Eliahou.
  \newblock The $3x+1$ problem: New lower bounds on nontrivial cycle lengths.
  \newblock {\em Discrete Mathematics}, 118(1--3):45--56, 1993.

  \bibitem{Garner-1981-1-OCA}
  Lynn~E. Garner.
  \newblock On the collatz $3n+1$ algorithm.
  \newblock {\em Proceedings of the American Mathematical Society}, 82(1):19--22,
    May 1981.
  \end{thebibliography}
\fi
\cleardoublepage

\end{document}
